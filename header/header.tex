\begin{center}
 \textbf{MỞ ĐẦU}
\end{center}
\par
\textbf{Deep Learning} là một thuật toán dựa trên một số ý tưởng từ não bộ tới việc tiếp thu nhiều tầng biểu đạt, cả cụ thể lẫn trừu tượng, qua đó làm rõ nghĩa của các loại dữ liệu. Deep Learning được ứng dụng trong nhận diện hình ảnh, nhận diện giọng nói, xử lý ngôn ngữ tự nhiên. Với dữ liệu khổng lồ hiện nay thì deep learning được ứng dụng vào rất nhiều các bài toán nhận dạng và cho thấy tính hiệu quả, độ chính xác cao so với các phương pháp truyền thống. \par
Những năm gần đây, ta đã chứng kiến được nhiều thành tựu vượt bậc trong ngành Thị giác máy tính (Computer Vision). Các hệ thống xử lý ảnh lớn như Facebook, Google hay Amazon đã đưa vào sản phẩm của mình những chức năng thông minh như nhận diện khuôn mặt người dùng, phát triển xe hơi tự lái hay drone giao hàng tự động. \par
Mạng nơ-ron tích chập (Convolutional Neural Network - CNN) là một trong những mô hình deep learning tiên tiến giúp chúng ta xây dựng được những hệ thống thông minh với độ chính xác cao như hiện nay. Trong khóa luận này, tôi đi vào nghiên cứu về mạng nơ-ron cũng như mạng nơ-ron tích chập, ý tưởng của mô hình CNN trong phân lớp ảnh (Image Classification), và áp dụng trong việc xây dựng hệ thống phân loại biển báo giao thông.\par
Nội dung khóa luận gồm 4 chương:
\begin{itemize}
\item[] \textbf{Chương 1: Tổng quan về bài toán phân loại biển báo giao thông.} Tại chương này tôi sẽ trình bày khái quát về bài toán phân loại biển báo và ứng dụng của nó trong tương lai.
\item[] \textbf{Chương 2: Cở sở toán học.} Ở chương này tôi nhắc lại một số kiến thức cơ bản để tiếp cận với cách hoạt động của mạng nơ-ron dễ dàng hơn.
\item[] \textbf{Chương 3: Mạng nơ-ron và mạng nơ-ron tích chập.} Chương này sẽ trình bày cấu trúc, thành phần và cách hoạt động của mạng nơ-ron và mạng nơ-ron tích chập.
\item[] \textbf{Chương 4: Ứng dụng mạng nơ-ron tích chập vào bài toán phân loại biển báo giao thông.} Tại chương này, tôi sẽ trình bài cách áp dụng mạng nơ-ron cho bài toán phân loại biển báo giao thông.
\end{itemize}